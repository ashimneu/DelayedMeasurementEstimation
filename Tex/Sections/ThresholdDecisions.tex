
\blue
Traditional approaches  to detect outliers are based on a threshold test:
\begin{equation} \label{eqn:threshold_test}
	b_{y_i} = \left\{
	\begin{array}{ll}
		0, & \mbox{ when }{|r_i|} \ge \lambda\sigma_{r_i}\\
		1, & \mbox{ when }{|r_i|} <   \lambda\sigma_{r_i}
	\end{array}
	\right.
\end{equation}
where $\lambda>0$ is the decision threshold, 
	$\sigma_{r_i}^2= {\Boldh_i \, \BoldP \, \Boldh_i^\top + \sigma_{i}^2}$ 
is the covariance of residual $r_i$ (see e.g.: \cite{fisher1992ical,neyman1933testing,frank1997survey,willsky1976survey,patton1994review}).
The residual and predicted measurement are computed as
\begin{equation} \label{eqn:residual} 
\Boldr =  \Boldy - \hat{\Boldy} 
~~	\mbox{ and } ~~	
\hat{\Boldy} = \BoldH \hat{\Boldx}^-. 
\end{equation}
In this approach $\Boldb_{\Boldx} = \1$.
The optimal estimate $\hat{\Boldx}_\Boldb$ and its posterior information $\BoldJ_{\Boldb}^{+}$ for the are  calculated using numeric methods theoretically equivalent to  eqn. (\ref{eqn:Xleastsquares}) and eqn. (\ref{eqn:PosteriorInfoMatrix}).

This threshold test approach requires the designer to select a decision threshold  $\lambda$. 
Assuming the prior is accurate, this threshold determines the probabilities of missed detections and false alarms. 
If an outlier is missed, the outlier measurement will affect the estimate mean $\hat{\Boldx}_\Boldb$ and information matrix $\BoldJ_{\Boldb}^+$; therefore, all subsequent decisions that use  $\Boldb_{\Boldx} = \1$ will be invalid due to being based on an invalid prior.

\black
