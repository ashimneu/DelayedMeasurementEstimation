%\tikzset{
%	treenode/.style = {align=center, inner sep=0pt, text centered,
%		font=\sffamily},
%	arn_n/.style = {treenode, circle, white, font=\sffamily\bfseries, draw=black,
%		fill=black, text width=1.5em},% arbre rouge noir, noeud noir
%	arn_r/.style = {treenode, circle, black, draw=black, 
%		text width=1.5em, very thick},% arbre rouge noir, noeud rouge
%	arn_x/.style = {treenode, rectangle, draw=black,
%		minimum width=0.5em, minimum height=0.5em}% arbre rouge noir, nil
%}
%
%%\begin{wrapfigure}{l}{8cm}
%	\begin{tikzpicture}[->,>=stealth',level/.style={sibling distance = 3cm/#1,
%			level distance = 1.5cm}] 
%		\node [arn_r] {$N_0$}	
%		child{ node [arn_r] {$N_1$} 			
%			edge from parent node[above left] {$b_{y_1} = 0$}
%		}
%		child{ node [arn_r] {$N_2$}
%			child{ node [arn_r] {$N_3$} 
%				edge from parent node[above left] {$b_{y_2} = 0$}
%			}
%			child{ node [arn_r] {$N_4$}
%				edge from parent node[above right] {$b_{y_2} = 1$}
%			}
%			edge from parent node[above right] {$b_{y_1} = 1$}
%		}; 
%	\end{tikzpicture}
%%	\caption{Branch \& Bound algorithm Tree}
%%\end{wrapfigure}

\begin{align} \label{prob:RAPS_cost_constraints}
	\left.
	\begin{array}{rl}
		\text{P}_0: & \underset{\Boldb}{\text{min}} ~ C(\Boldb) \\ 
		& \text{subject to:} \\
		& \sum_{i=1}^m \sigma_i^{-2} b_{y_i} 
		\Boldh_i^\top \circ \Boldh_i^\top \ge \Delta \Boldj \\
		& b_i \in \{0,1\}
	\end{array}
	\right\}
\end{align}
where $\Delta \Boldj = diag(\BoldJ_l) - diag(\BoldJ^-)$ and $\Boldh_i^\top \circ \Boldh_i^\top$ represents the element-wise product of vector $\Boldh_i^\top$ with itself.
The integer linear programming (ILP) problem P$_0$ consists of a cost function and $m$ inequality constraints that are linear in the binary selection variables $\Boldb_{\Boldy}$. When the integer constraint on $b_{y_i}$ is relaxed to an interval in real number line, i.e. $b_{y_i} \in [0,1]$, following linear programming (LP) problem is obtained:	
\begin{align} \label{prob:RAPS_cost_constraints_relaxed2}
	\left.
	\begin{array}{rl}
		^\text{r}\text{P}_{0}: &
		\text{minimize} ~ \sum_{i=1}^m b_{y_i} r^2_i \sigma^{-2}_i \\ 
		& \text{subject to:} \\
		& \sum_{i=1}^m \sigma_i^{-2} b_{y_i} 
		\Boldh_i^\top \circ \Boldh_i^\top \ge \Delta \Boldj \\
		& b_{y_i} \in [0,1]
	\end{array}
	\right\}
\end{align}

The LP optimization problem in eqn. (\ref{prob:RAPS_cost_constraints_relaxed}) is can be easily solved using a simplex method. The feasibility set

%\begin{wrapfigure}{l}{3.0cm}
%	\begin{tikzpicture}[every node/.style={draw=black,thick,circle,inner sep=2pt}]
%		\node[circle,draw](z){$N_0$}
%		child{ 
%			node[circle,draw]{$N_1$} 
%			child{node[circle,draw] {$N_3$}} 
%			child{node[circle,draw] {$N_4$}} 
%		}
%		child{ 
%			node[circle,draw]{$N_2$}
%			edge from parent node [above right]
%			{$x$}
%		};
%	\end{tikzpicture}
%	\caption*{B\&B Tree}
%\end{wrapfigure}



At each iteration of the Branch and Bound algorithm, $C^*$ and $\Boldb_{\Boldy}^*$  will be referred to as $incumbent~cost$ and $incumbent~solution$, respectively, and their initial values are assigned as $C^* = \infty$ and $\Boldb_{\Boldy}^*$ = [1,...,1]$^\top$. 
A set of unexplored nodes is denoted by L. 
Every node in the branch and bound tree corresponds to a binary integer programming (IP) problem and the root node corresponds to the IP problem in eqn. (\ref{prob:RAPS_cost_constraints}).

\begin{enumerate}
	\item A node is selected from the list of unexplored nodes L.
	\item \label{step:BB_relaxed_constraint} The integer constraint on $b_{y_i}$ is relaxed to $b_{y_i} \in [0,1]$ which yields the corresponding relaxed constrained optimization problem:	
	\begin{align} \label{prob:RAPS_cost_constraints_relaxed}
		\left.
		\begin{array}{rl}
			\text{P}_{0r}: &
			\text{minimize} ~ \sum_{i=1}^m b_{y_i} r^2_i \sigma^{-2}_i \\ 
			& \text{subject to:} \\
			& \sum_{i=1}^m \sigma_i^{-2} b_{y_i} 
			\Boldh_i^\top \circ \Boldh_i^\top \ge \Delta \Boldj \\
			& b_{y_i} \in [0,1]
		\end{array}
		\right\}
	\end{align}
	The optimization problem in eqn. (\ref{prob:RAPS_cost_constraints_relaxed}) is a linear programming problem which can be easily solved using a simplex method.
	
	\item \label{step:BB_branching} The solution to eqn. (\ref{prob:RAPS_cost_constraints_relaxed}) can consist of non-integer values for the variable $b_{y_i}$ which is unacceptable for the original integer problem in P$_0$. Among the non-integer variables, one of the $\{b_{y_i}\}_{i=1}^m$ is picked by a heuristic function. With  no loss of generality let $b_{y_1}$ be the variable picked by the heuristic function.
	\item From the root node $N_0$ generate two child nodes, $N_1$ and $N_2$. Node $N_1$ consists same IP problem as the parent node $N_0$ but, $b_{y_1}$ = 1. On the other hand, node $N_2$ consists same IP problem as the parent node $N_0$ where, $b_{y_1}$ = 0.
		
	\item The unexplored children nodes $N_1$ and $N_2$ are added to the set L.
	
	\item  One of the unexplored nodes from set L is picked by using a predefined search strategy. Let the node $N_1$ be picked for further exploration by the algorithm.
	
	\item IP problem of the node $N_1$ is converted to a LP problem by relaxing the integer constraint by using the method described in Step (\ref{step:BB_relaxed_constraint}). 
	Solution to the LP problem $\Boldb_{\text{LP}}$ is obtained and the corresponding cost $C_{\text{LP}}$ is computed.
\end{enumerate}

\begin{algorithm}[t]  
	\setlength{\belowdisplayskip}{-5pt} 
	\setlength{\belowdisplayshortskip}{-5pt}
	\setlength{\abovedisplayskip}{0pt} 
	\setlength{\abovedisplayshortskip}{-1pt}
	\caption{Branch \& Bound Algorithm} 
	\label{alg:RAPS_BnB}                   
	\begin{algorithmic}[1] 
		\Require $\BoldH$, $\BoldR$, $\BoldP$, $\hat{\Boldx}^-$.
		\Ensure $\Boldb_{\Boldy}^*$, $C^*$.
		\State Create a set of unexplored nodes, $L=\{\}$;
		\State Add root node $N_0$ in set L;
		\State Set $incumbent~cost$ $C^* = \infty$ 
		\State Set $incumbent~solution$ $\Boldb_{\Boldy}^*$ = [1,..,1]$^\top$.
		\While{set of unexplored nodes L $\ne$ \{\}}
		\State Use a search strategy to select a node $N_j$ from L.
		\State Remove node $N_j$ from L and assign it as active node; 
		\State Get corresponding problem P$_j$;
		\State For ILP problem P$_j$ get the relaxed problem $^\text{r}$P$_{j}$;
		\State Solve $^\text{r}$P$_{j}$ using LP solver such as simplex method;
		\State Let $C_{\text{LP}}$ and $\Boldb_{\text{LP}}$ denote cost and solution of $^\text{r}$P$_{j}$;
		\Statex \quad\quad \textit{Case 1.} If $C_{\text{LP}} \ge $ $C^*$ prune current node $N_j$.
		\Statex \quad\quad \textit{Case 2.} If $C_{\text{LP}} < $ $C^*$ and $\Boldb_{\text{LP}}$ is feasible for ILP problem
		\Statex $~~~~~~~~~~~~~~~~$ P$_j$ then $C^*$ = $C_{\text{LP}}$ and $\Boldb_y^*$ = $\Boldb_{\text{LP}}$; Prune $N_j$;
		\Statex \quad\quad \textit{Case 3.} If $C_{\text{LP}} < $ $C^*$ and $\Boldb_{\text{LP}}$ is infeasible for P$_j$ then 
		\Statex \qquad \qquad \qquad branch at nodes $N_j$ and add its children nodes  
		\Statex \qquad \qquad \qquad to set L;

		\EndWhile 		
	\end{algorithmic}
\end{algorithm}

\newpage
\noindent
{\textit{Proof}}\\
The minimization of the cost $C(\Boldx,\Boldb_{\Boldy})$ given in eqn. (\ref{eqn:RAPS_cost_linearb}) constrained by the linear inequalities in eqn. (\ref{eqn:RAPS_linear_constraint}) can be accomplished iteratively. 
Each iteration, $\ell$, consists of two parts.

\textit{Part 1}
In the first part, $C(\Boldx,\Boldb_{\Boldy})$ is minimized with respect to $\Boldb_{\Boldy}$ while fixing the value of $\Boldx$ to its estimate from the preceding iteration, i.e. $\Boldx = \hat{\Boldx}^{\ell-1}$. 
For $\ell = 1$, $\Boldx$ is fixed to the state prior $\hat{\Boldx}^-$.
After the binary constrained on $b_i$ is relaxed, the corresponding minimization problem can be represented as
\begin{align} \label{prob:RAPS_cost_constraints_relaxed3}
	\left.
	\begin{array}{rl}
		^\text{r}\text{P}_{\ell}: &
		\norm{ \Boldx^{\ell-1} - \hat{\Boldx}^-}_{\BoldP}^2 
		+   \sum_{i=1}^{m} \frac{b_{y_i}}{\sigma_i^2} (\Boldh_i 
		\Boldx^{\ell} - y_i)^2 \\ 
		& \text{subject to:} \\
		& \sum_{i=1}^m \sigma_i^{-2} b_{y_i} 
		\Boldh_i^\top \circ \Boldh_i^\top \ge \Delta \Boldj \\
		& b_{y_i} \in [0,1]
	\end{array}
	\right\}
\end{align}

The problem in eqn. (\ref{prob:RAPS_cost_constraints_relaxed3}) consists of cost and constraints that are linear in $\Boldb_{\Boldy}$ therefore it can be minimized using the branch and bound search approach described in Algorithm (\ref{alg:RAPS_BnB}). This yields the  cost $C(\Boldx^{\ell-1},\Boldb^{\ell}_{\Boldy})$ and a binary selection vector, $\Boldb^{\ell}_{\Boldy}$, which is integer feasible.

\textit{Part 2}
The newly found selection vector $\Boldb^{\ell}_{\Boldy}$ is substituted in the cost which is then minimized with respect to $\Boldx$. 
This yields the solution $\hat{\Boldx}^{\ell}$ and the minimum $C(\hat{\Boldx}^{\ell},\Boldb^{\ell}_{\Boldy})$. 
$\hat{\Boldx}^{\ell}$ is equivalent to the solution of eqn. (\ref{eqn:KF_info_msr_selected}) when $\BoldPhi(\Boldb_{\Boldx}) = \BoldI$ and $\Boldb_{\Boldy} = \Boldb^{\ell}_{\Boldy}$. 
Note that $C(\hat{\Boldx}^{\ell},\Boldb^{\ell}_{\Boldy}) \le C(\hat{\Boldx}^{\ell-1},\Boldb^{\ell}_{\Boldy})$ because $\hat{\Boldx}^{\ell}$ yields the lowest possible cost when $\Boldb_{\Boldy}$ is fixed to $\Boldb^{\ell}_{\Boldy}$.
$\hat{\Boldx}^{\ell}$ is substituted into the cost for the minimization described in the \textit{Part 1} for the iteration $\ell+1$.

\noindent
The iteration is stopped when $\Boldb^{\ell}_{\Boldy} = \Boldb^{\ell-1}_{\Boldy}$ since the cost cannot be further minimized.

\clearpage


