\textit{1. Pseudorange measurement} \\
This sections redefines observation equations to account for the presence of
pseudorange measurements from navigation satellite systems such as Galileo and BeiDou, along with
GPS satellite system.

The receiver clock state vector is redefined as $\Boldx_c = [t_r, \, \theta_{E}, \, \theta_{B}, \, b_r]^\top$
where $\theta_E$ and $\theta_B$ are inter-system clock biases of Galileo and BeiDou satellite systems, respectively.

The measurement matrix $\BoldH$ is redefined as
\begin{equation}
	\BoldH = 
	\begin{bmatrix}
		\BoldH_G  \\ \BoldH_E  \\ \BoldH_B 
	\end{bmatrix}
\end{equation}

When the GNSS observations in $\Boldy$ consist of pseudorange measurements only, the measurement matrices for GPS, Galileo and BeiDou systems are defined as
\begin{align}
	\BoldH_G &= \begin{bmatrix}		
		\mathbfcal{H} & \0_{m \times 3} & \0_{m \times 3} & \1_m & \0_m & \0_m & \0_m		
	\end{bmatrix}, \nonumber \\
	\BoldH_E &= \begin{bmatrix}
		\mathbfcal{H} & \0_{m \times 3} & \0_{m \times 3} & \1_m & \1_m & \0_m  & \0_m	
	\end{bmatrix}, \nonumber \\
	\BoldH_B &= \begin{bmatrix}		
		\mathbfcal{H} & \0_{m \times 3} & \0_{m \times 3} & \1_m & \0_m & \1_m  & \0_m	
	\end{bmatrix},
\end{align}
respectively.

\noindent
The continuous-time description of the clock model is 
\begin{align}
	\dot{\Boldx}_c(t) \doteq 
	\begin{bmatrix}
		0 & 0 & 0 & 1 \\
		0 & 0 & 0 & 0 \\ 
		0 & 0 & 0 & 0 \\  
		0 & 0 & 0 & -\lambda_{t_r}
	\end{bmatrix} 
	\Boldx_c(t) +
	\begin{bmatrix} 
		0 & 0 & 0 & 0 \\ 
		0 & 1 & 0 & 0 \\ 
		0 & 0 & 1 & 0 \\ 
		0 & 0 & 0 & 1 \\ 
	\end{bmatrix} 
	\Boldw_c(t)
\end{align}
where $\Boldw_c(t)$ is modeled as Gaussian white noise with power spectral density $P_c = \sigma_c^2$. 
The corresponding discrete-time description of the clock model is 
\begin{align}
	\BoldF_c &= 
	\begin{bmatrix}
		1 & 0 & 0 & \beta_1 \\ 
		0 & 1 & 0 & 0 \\
		0 & 0 & 1 & 0 \\
		0 & 0 & 0 & \beta_2
	\end{bmatrix}, ~ \\
	\BoldGamma_c & \approx
	\begin{bmatrix}
		(T^3 / 3)^{\frac{1}{2}} & 0 & 0 & 0 \\
		0 & \sqrt{T} & 0 & 0 \\
		0 & 0 & \sqrt{T} & 0 \\
		0 & 0 & 0 & \sqrt{T}
	\end{bmatrix}, \\ 
	\BoldQ_{c} &= 
	\begin{bmatrix}
		\sigma_{t_r}^2 & 0 & 0 & 0 \\
		0 & \sigma_{\theta_E}^2 & 0 & 0 \\
		0 & 0 & \sigma_{\theta_B}^2 & 0 \\ 
		0 & 0 & 0 & \sigma_{b_r}^2
	\end{bmatrix}
\end{align}
where $\beta_1 = (1 - \exp^{-\lambda_c T})/ \lambda_c$ and $\beta_2 = \exp^{-\lambda_c T}\!\!.$
The approximations in $\BoldGamma_v$ and $\BoldGamma_c$ yield the correct diagonal of the discrete-time noise covariance matrix but approximates the off-diagonal terms.

\begin{table}[h!]
	\centering
	\begin{tabular}[h]{|c|c|c|c|c|c|c|c|c|}
		\hline
		Quantity & 	$\lambda_a$ & $\sigma_a^2$ & $\lambda_{t_r}$ & $\sigma_{t_r}^2$ & $\sigma_{\theta_E}^2$ & $\sigma_{\theta_B}^2$ & $\sigma_{b_r}^2$ \T \\ \hline
		value	 &  0.1 & 1.0 & 0.01 & 0.001 & 0.1 & 0.1 & 0 \T \\ \hline
		unit	 &  $s^{-1}$ & $s^{-1}$ &  $m^2s^{-5}$ & $m^2s^{-3}$ & $m^2s^{-3}$ & $m^2s^{-3}$ & $m^2s^{-3}$ \T \\ \hline				
	\end{tabular}
	\caption{Parameters of the continuous-time Markov process.}
	\label{table:error_std_prediction}
\end{table}