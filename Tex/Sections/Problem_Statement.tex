
%The overall state estimation process involves two steps, time propagation and measurement update. 
%While this article includes a linearized time propagation for the completeness of the estimation process, the prime focus shall be on the measurement update portion. 

Recursive Bayesian estimation consists of time propagation for prediction of the state and measurement update to correct the predicted state.

The continuous time propagation model can be represented as
\begin{equation}
	\label{eqn:StatePropModel}
	\Boldx_{k+1} = \Boldf (\Boldx_k, \Boldu_k ) + \Boldomega_k
\end{equation}
where $\Boldx_k \in\mathbb{R}^n$ represents the state vector at discrete-time $k$, $\Boldu_k$ is the control input and $\Boldw_k$ is zero-mean, white, Gaussian process noise with covariance matrix $\BoldQ_k$. $\Boldf (\cdot)$ represents the relation for propagation of the system between successive time steps.
The measurement models are represented as
\begin{align} 
	\label{eqn:DelayedMeasurementModel}
	\Boldz_k &= \Boldh ( \Boldx_{k}, \Boldb_{k}, \tau_k ) + \Boldzeta_k \\
	\label{eqn:UndelayedMeasurementModel}
	\Boldy_k &= \Boldc ( \Boldx_k ) + \Boldeta_k,
\end{align}
where $\Boldz_k$ is a vector of delayed measurements affected by bias $\Boldb_k$, $\Boldy_k$ is a vector of measurements unaffected by the delay and $\tau_k = N$ is number of delay time steps.
$\Boldh(\cdot)$ and $\Boldc(\cdot)$ represent the relation between the system state and the measurements $\Boldz_k$ and $\Boldy_k$, respectively.
$\Boldzeta_k$ and $\Boldeta_k$ are zero-mean, white, Gaussian measurement noises with covariance matrices $\BoldR_k$ and $\BoldSigma_k$, respectively.
The covariance matrices $\BoldSigma_k$ and $\BoldR_k$ are assumed to be invertible and diagonal\footnote
{\label{ftnt:R_assumption}
	Note that there is no restriction attached to this assumption. The solution can be used for any  covariance matrix by using the transformation $\Boldy' =\BoldSigma_R  \, \Boldy$ with $\BoldR^{-1} = \BoldSigma_R \, \BoldSigma_R^\top$, the measurement model for $\Boldy'$ is:
	$$\Boldy'=\BoldH' \,\Boldx+\Boldeta' \mbox{ where } \BoldH' = \Sigma_R  \, \BoldH \mbox{ , } \Boldeta' \sim \mathcal{N}(\0,\,\BoldI).$$
}
, which can be written as $\BoldSigma = \sum_{i=1}^{m_d} \rho_{i}^2 \, \Bolde_i  \,{\Bolde_i}^\top$ and $\BoldR = \sum_{i=1}^{m_u} \sigma_{i}^2 \, \Bolde_i  \,{\Bolde_i}^\top$, respectively, where $\Bolde_i$ is the $i^{th}$ column of the identity matrix $\BoldI \in \mathbb{R}^{m \times m}$ .
The prior and measurement noise are independent.

The linear discrete-time form of the time propagation model of eqn. (\ref{eqn:StatePropModel}) can be represented as
\begin{align}
	\label{eqn:LinearStatePropModel}
	\Boldx_{k+1} &\doteq \BoldF \Boldx_k + \BoldGamma \Boldw_k.
\end{align}
The linear representations of the measurement models of eqns. (\ref{eqn:DelayedMeasurementModel}) and (\ref{eqn:UndelayedMeasurementModel}) are
\begin{align} \label{eqn:LinearDelayedMeasurementModel}
	\Boldz_k &\doteq \BoldH_{k-N} \Boldx_{k-N} + \Boldb_{k-N} +\Boldzeta_k \\
	\label{eqn:LinearUndelayedMeasurementModel}
	\Boldy_k &\doteq \BoldC_k \Boldx_k + \Boldeta_k,
\end{align}
where $\BoldH_{k-N} \in \mathbb{R}^{m_d \times n}$ is the measurement matrix that relates the delayed measurement $\Boldz_{k}$ to the state at time $k-N$, i.e. $\Boldx_{k-N}$. 
$\BoldC_k\in \mathbb{R}^{m_u \times n}$ is the measurement matrix that relates undelayed measurement $\Boldz_k$ to the state at time $k$, i.e. $\Boldx_k$ .

The measurement update at time $k$ involves a state prior which is modeled as a Gaussian probability distribution $\Boldx_k \sim \mathcal{N}(\hat{\Boldx}_k^-,\BoldP_k^-)$, where the mean $\hat{\Boldx}_k^-$ and error covariance matrix $\BoldP_k^-$ are provided by the time propagation step.