Recursive state estimation consists of two steps:
(1) time propagation to advance the state estimate from time $t_k$ to $t_{k+1}$, and
(2) measurement update which uses the measurement to correct the predicted state.

The discrete-time propagation model is represented as
\begin{equation}
	\label{eqn:StateProp}
	\Boldx_{k+1} = \Boldf (\Boldx_k, \Boldu_k ) + \Boldomega_k
\end{equation}
where 
$\Boldx_k \in\mathbb{R}^n$ represents the state vector at discrete-time $k$,
$\Boldu_k \in \mathbb{R}^p$ is a known deterministic input, and 
$\Boldw_k \in \mathbb{R}^n$ is zero-mean, white, Gaussian process noise with covariance matrix $\BoldQ_k \in \mathbb{R}^{n \times n}$. 
The function $\Boldf (\cdot)$ describes the propagation of the state between successive time steps.
The measurement models are
\begin{align} 
	\label{eqn:UndelayedMeasurement}
	\Boldy_k &= \Boldc ( \Boldx_k ) + \Boldeta_k, \\
	\label{eqn:DelayedMeasurement}
	\Boldz_k &= \Boldh ( \Boldx_{k}, \Boldb_{k}, \tau_k ) + \Boldzeta_k
\end{align}
where 
$\Boldy_k \in \mathbb{R}^{m_u}$ is a vector of measurements unaffected by the delay;
and, 
$\Boldz_k \in \mathbb{R}^{m_d}$ is a vector of measurements affected by measurement bias $\Boldb_k \in \mathbb{R}^{m_d}$  and delayed by \red $\tau_k = N \in \mathbb{R}$ is number of delay time steps;\todo{AN: I do not quite understand. I interpret the symbol $N$ as and integer, but you state that it is real. Also, there is not $T$. Are you assuming that the measurement is delayed by an integer number of sample periods?  The two figures indicate that the delay is a time varying real number. Please clarify.} \black
$\Boldc(\cdot)$ and $\Boldh(\cdot)$ are functions describing the relationship between the measurements $\Boldy_k$ and $\Boldz_k$ and the system state, respectively; and,
$\Boldzeta_k \in \mathbb{R}^{m_d}$ and $\Boldeta_k \in \mathbb{R}^{m_u}$ are zero-mean, white, Gaussian measurement noise vectors with covariance matrices $\BoldR_k \in \mathbb{R}^{m_d}$ and $\BoldSigma_k \in \mathbb{R}^{m_u}$, respectively.
The covariance matrices $\BoldSigma_k$ and $\BoldR_k$ are assumed to be invertible and diagonal\footnote
{\label{ftnt:R_assumption}
	Note that there is no restriction attached to this assumption. The solution can be used for any  covariance matrix by using the transformation $\Boldy' =\BoldSigma_R  \, \Boldy$ with $\BoldR^{-1} = \BoldSigma_R \, \BoldSigma_R^\top$, the measurement model for $\Boldy'$ is:
	$$\Boldy'=\BoldH' \,\Boldx+\Boldeta' \mbox{ where } \BoldH' = \Sigma_R  \, \BoldH \mbox{ , } \Boldeta' \sim \mathcal{N}(\0,\,\BoldI).$$
}
, which can be written as $\BoldSigma = \sum_{i=1}^{m_d} \rho_{i}^2 \, \Bolde_i  \,{\Bolde_i}^\top$ and $\BoldR = \sum_{i=1}^{m_u} \sigma_{i}^2 \, \Boldf_i  \,{\Boldf_i}^\top$, respectively. $\Bolde_i$ and $\Boldf_i$ are the $i^{th}$ column of the identity matrices $\BoldI \in \mathbb{R}^{m_d \times m_d}$ and $\BoldI \in \mathbb{R}^{m_u \times m_u}$, respectively.

At the initial time $t_0$, the initial state has the distribution $\Boldx_0\sim N(\hat\Boldx_0,\BoldP_0)$, where the covariance matrix $\BoldP_0$ is positive definite.
The prior and measurement noise are assumed to be independent.

The linear discrete-time form of the time propagation model of eqn. (\ref{eqn:StateProp}) can be represented as
\begin{align}
	\label{eqn:LinearStateProp}
	\Boldx_{k+1} &\doteq \BoldF \, \Boldx_k + \BoldGamma \, \Boldomega_k.
\end{align}
The linear representations of the measurement models of eqns. (\ref{eqn:DelayedMeasurement}) and (\ref{eqn:UndelayedMeasurement}) are
\begin{align} \label{eqn:LinearDelayedMeasurement}
	\Boldz_k &\doteq \BoldH_{k-N} \, \Boldx_{k-N} + \Boldb_{k-N} +\Boldzeta_k \\
	\label{eqn:LinearUndelayedMeasurement}
	\Boldy_k &\doteq \BoldC_k \, \Boldx_k + \Boldeta_k,
\end{align}
where $\BoldH_{k-N} \in \mathbb{R}^{m_d \times n}$ is the measurement matrix that relates the delayed measurement $\Boldz_{k}$ to the state at time $k-N$ (i.e., $\Boldx_{k-N}$) and 
$\BoldC_k\in \mathbb{R}^{m_u \times n}$ is the measurement matrix that relates undelayed measurement $\Boldz_k$ to the state at time $k$ (i.e., $\Boldx_k$).

The measurement update at time step $k$ relies on a  prior state estimate which is modeled as a Gaussian probability distribution $\Boldx_k \sim \mathcal{N}(\hat{\Boldx}_k^-,\BoldP_k^-)$, where the mean $\hat{\Boldx}_k^-$ and error covariance matrix $\BoldP_k^-$ are provided by the time propagation step.

