
%The overall state estimation process involves two steps, time propagation and measurement update. 
%While this article includes a linearized time propagation for the completeness of the estimation process, the prime focus shall be on the measurement update portion. 

Recursive Bayesian estimation consists of time propagation for prediction of the state and measurement update to correct the predicted state.
When there is delay in the time a sensor provides a measurement and the time it is made available for the measurement update, the estimation process need to be address the delay before incorporating the measurement into the update.
This article describes an approach to address the delay for a discrete-time estimation model.

Let $\Boldx_k \in\mathbb{R}^n$ represent the state vector at discrete-time $k$.
The measurement update at time $k$ involves a state prior and a measurement vector $\Boldy_k\in\mathbb{R}^m$.
%
The prior for the state $\Boldx_k$ is modeled as a Gaussian probability distribution $\Boldx_k \sim \mathcal{N}(\hat{\Boldx}_k^-,\BoldP_k^-)$, where the mean $\hat{\Boldx}_k^-$ and error covariance matrix $\BoldP_k^-$ are provided by the time propagation step.
  
The delayed measurement vector at time $k$ is:
\begin{align} \label{eqn:MeasurementModel}
	\Boldz_{k} = \BoldC_{k-N} \, \Boldx_{k-N} + \Boldeta_k
\end{align}
where $N$ is delay time steps, $\BoldC_{k-N}\in \mathbb{R}^{m \times n}$ is the measurement matrix and $\Boldeta_k \sim \mathcal{N}(\0,\,\BoldR)$ is white Gaussian measurement noise.

The undelayed measurement vector at time $k$ is:
\begin{align} \label{eqn:MeasurementModel}
\Boldy_k = \BoldH_k \, \Boldx_k + \Boldeta_k
\end{align}
where $\BoldH_k \in \mathbb{R}^{m \times n}$ is the measurement matrix and $\Boldeta_k \sim \mathcal{N}(\0,\,\BoldR)$ is white Gaussian measurement noise. The covariance matrix $\BoldR$ is assumed to be invertible and diagonal\footnote
{\label{ftnt:R_assumption}
	Note that there is no restriction attached to this assumption. The solution can be used for any  covariance matrix by using the transformation $\Boldy' =\BoldSigma_R  \, \Boldy$ with $\BoldR^{-1} = \BoldSigma_R \, \BoldSigma_R^\top$, the measurement model for $\Boldy'$ is:
	$$\Boldy'=\BoldH' \,\Boldx+\Boldeta' \mbox{ where } \BoldH' = \Sigma_R  \, \BoldH \mbox{ , } \Boldeta' \sim \mathcal{N}(\0,\,\BoldI).$$
} 
which can be written as $\BoldR = \sum_{i=1}^{m} \sigma_{i}^2 \, \Bolde_i  \,{\Bolde_i}^\top$, where $\Bolde_i$ is the $i^{th}$ column of the identity matrix $\BoldI \in \mathbb{R}^{m \times m}$ .
The prior and measurement noise are independent.

Some of the measurements will be affected by outliers.
Outlier measurements are those that are very unlikely given the model stated in eqn. (\ref{eqn:MeasurementModel}). 
When the $i$-th measurement is affected by an outlier, that measurement could be considered to be modeled by
\begin{equation}\label{eqn:MeasurementModel_s}
	y_i = \Boldh_i \, \Boldx + \eta_i +s_i,
\end{equation} where $s_i$ represents the outlier and $\Boldh_i$ is the $i$-th row of $\BoldH_k$; however, 
there is usually no known model for the outlier $s_i$.




