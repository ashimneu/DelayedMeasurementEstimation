At time step $k$, the system state vector is $\Boldx = [\,\Boldp^\top \dss,  \, \Boldv^\top \dss, \, \Bolda^\top \dss, t_r, \beta_r, \Boldb^\top \dss, \Boldd^\top ]^\top \in \mathbb{R}^{n_s} $, where $\Boldp$, $\Boldv$, $\Bolda$ $\in \mathbb{R}^3$ represent the rover position, velocity and acceleration, respectively.
The use of subscript to represent time step has been dropped for current section. 
%Instead it shall be used to denote a certain portion of the variable corresponding to the state variables represented by the   $\Boldx$.
The PVA state vector is $\Boldx_v = [\,\Boldp^\top \dss,  \, \Boldv^\top \dss, \, \Bolda^\top]$.
The receiver clock state vector is $\Boldx_c = [t_r, \beta_r]^\top$, where $t_r$, $\beta_r$ $\in \mathbb{R}$ are the receiver clock bias and clock drift, respectively. 
The bias state vector is $\Boldx_b = [ \Boldb^\top \dss, \Boldd^\top ] \in \mathbb{R}^{m_d}$ where $\Boldb$ is the vector of bias and $\Boldd$ is the vector of bias drift. 
Therefore, $n_s = 11 + m_d$. $\BoldF \in \mathbb{R}^{n_s \times n_s}$ is the state transition matrix.
$\Boldw \sim \mathcal{N}(\0,\,\BoldQ)$ is the process noise vector which is assumed to have a Gaussian distribution. $\BoldGamma \in \mathbb{R}^{n_s \times n_s}$ is the corresponding noise matrix. $\tau$ and $T$ are used to denote the time increments between successive state propagation and measurement update steps. 
Since the propagation is done at higher rate compared to measurement update, $\tau \ll T$.
When the GNSS pseudorange measurements are used the observation matrix $\BoldC$ is
\begin{equation}
	\BoldC = \begin{bmatrix}		
		\mathbfcal{C} & \0_{m \times 3} & \0_{m \times 3} & \1_{m \times 1} & \0_{m \times 1} \\		
	\end{bmatrix}
\end{equation}
where $\mathbfcal{C} = [\, \mathfrak{C}^1 \, \mathfrak{C}^2 \, \dots \, \mathfrak{C}^m ]^\top$ and each $\mathfrak{C}^s$ represents a line-of-sight unit vector between satellite $s$ and the rover GNSS antenna. 
The symbols $\0_{a \times b}$ and $\1_{a \times b}$ are  matrices consisting of $a$ rows and $b$ columns whose all entries are $0$s and $1$s, respectively. 

The sub-components $\Boldx_v$, $\Boldx_c$ and $\Boldx_b$ propagate independently through time hence their respective components in the state space model are not coupled.
The matrices of the discrete-time state-space model are
\begin{align} 
	\BoldF = \begin{bmatrix} \BoldF_v \dss & \0_3 \dss & \0_3 \\ \0_3 \dss & \BoldF_c \dss & \0_3 \\ \0_3 \dss & \0_3 \dss & \BoldF_b \end{bmatrix} \dss, 
	\BoldGamma = \begin{bmatrix} \BoldGamma_v \dss & \0_3 \dss & \0_3 \\ \0_3 \dss & \BoldGamma_c \dss & \0_3 \\ \0_3 \dss & \0_3 \dss & \BoldGamma_b \end{bmatrix} \dss, 
	\BoldQ = \begin{bmatrix} \BoldQ_v \dss & \0_3 \dss & \0_3 \\ \0_3 \dss & \BoldQ_c \dss & \0_3 \\ \0_3 \dss & \0_3 \dss & \BoldQ_b \end{bmatrix}
\end{align}
%The time step of state propagation is defined as $T = 0.1$ seconds and the measurement update using GNSS pseudorange occur once every 1 second.

The continuous-time PVA vehicle model is
\begin{align}
	\dot{\Boldx}_v(t) \doteq 
	\begin{bmatrix}
		\0_3 & \BoldI_3 & \0_3 \\ \0_3 & \0_3 & \BoldI_3 \\ \0_3 & \0_3 & -\lambda_a \BoldI_3
	\end{bmatrix} 
	\Boldx_v(t) +
	\begin{bmatrix} \0_3 & \0_3 & \0_3 \\ \0_3 & \0_3 & \0_3 \\ \0_3 & \0_3 & \BoldI_3 \end{bmatrix} \Boldw_a(t)
\end{align}
where $\Boldomega_a(t)$ is modeled as Gaussian white noise with power spectral density $\BoldP_a = \sigma^2_a$. 
The corresponding discrete-time description of the PVA vehicle model is approximated as
\begin{align}
	\BoldF_v &= 
	\begin{bmatrix}
		\BoldI_3 & T\,\BoldI_3 & a_3 \, \BoldI_{3} \\
		\0_3 & \BoldI_3 & a_2 \, \BoldI_{3} \\
		\0_3 & \0_3 & a_1 \, \BoldI_{3}
	\end{bmatrix}, ~ \\
	\BoldGamma_v &\approx
	\begin{bmatrix}
		\big(T^5/20 \big)^{\frac{1}{2}} \, \BoldI_3 & \0_3 & \0_3\\
		\0_3 & \big(T^3 / 3\big)^{\frac{1}{2}} \, \BoldI_3 & \0_3 \\
		\0_3 & \0_{3 \times 3} & \sqrt{T} \, \BoldI_3
	\end{bmatrix}, \\ 
	\BoldQ_{v} &= 
	\begin{bmatrix} \0_3 & \0_3 & \0_3 \\ 
					\0_3 & \0_3 & \0_3 \\
		 			\0_3 & \0_3 & \sigma^2_a \, \BoldI_3 
	\end{bmatrix},
\end{align}
where $a_1 = \exp^{\lambda_a T}$, $a_2 = (1 - \exp^{-\lambda_a T})/ \lambda_a$, and $a_3 = (\lambda_a T - 1 + \exp^{-\lambda_a T})/ \lambda_a^2$.

The continuous-time description of the clock model is 
\begin{align}
	\dot{\Boldx}_c(t) \doteq 
	\begin{bmatrix}
		0 & 1 \\ 0 & -\lambda_c
	\end{bmatrix} 
	\Boldx_c(t) +
	\begin{bmatrix} 0 \\ 1 \end{bmatrix} \Boldw_c(t)
\end{align}
where $\Boldw_c(t)$ is modeled as Gaussian white noise with power spectral density $\BoldQ_c = \sigma_c^2$. 
The corresponding discrete-time description of the clock model is 
\begin{align}
	& \BoldF_c = 
	\begin{bmatrix}
		1 & \pi_1 \\ 0 & \pi_2
	\end{bmatrix}, ~
	\BoldGamma_c \approx
	\begin{bmatrix}
		(T^3 / 3)^{\frac{1}{2}} & 0 \\
		0 & \sqrt{T}
	\end{bmatrix}, ~  
	\BoldQ_{c} = 
	\begin{bmatrix}
		\sigma_{t_r}^2 & 0 \\ 0 & \sigma_{\beta_r}^2
	\end{bmatrix}
\end{align}
where $\pi_1 = (1 - \exp^{-\lambda_c T})/ \lambda_c$ and $\pi_2 = \exp^{-\lambda_c T}$.
The approximations in $\BoldGamma_v$ and $\BoldGamma_c$ yield the correct diagonal of the discrete-time noise covariance matrix but approximates the off-diagonal terms.

The continuous-time description of the measurement bias
\begin{align}
	\dot{\Boldx}_b(t) \doteq 
	\begin{bmatrix}
		\0_3 & \BoldI_3 \\ \0_3 & -\lambda_b \BoldI_3
	\end{bmatrix} 
	\Boldx_b(t) +
	\begin{bmatrix} \0_3 \\ \BoldI_3 \end{bmatrix} \Boldw_b(t)
\end{align}
where $\Boldw_b(t)$ is modeled as Gaussian white noise with power spectral density $\BoldQ_b = \sigma_b^2$. 
The corresponding discrete-time description of the clock model is 
\begin{align}
	& \dss \BoldF_b = 
	\begin{bmatrix}
		\BoldI_3 \dss & \chi_1 \BoldI_3 \\ \0_3 \dss & \chi_2 \BoldI_3
	\end{bmatrix} \ds,
	\BoldGamma_c \approx
	\begin{bmatrix}
		\sqrt{(\frac{T^3}{3})} \BoldI_{3} \dss & \0_3 \\
		0 \dss & \sqrt{T} \BoldI_3
	\end{bmatrix} \ds,  
	\BoldQ_{c} = 
	\begin{bmatrix}
		\0_3 \dss & \0_3 \\ \0_3 \dss & \sigma_{d}^2 \BoldI_3
	\end{bmatrix}
\end{align}
where $\chi_1 = (1 - \exp^{-\lambda_b T})/ \lambda_c$ and $\chi_2 = \exp^{-\lambda_b T}$.
The approximations in $\BoldGamma_v$, $\BoldGamma_c$ and $\BoldGamma_b$ yield the correct diagonal of the discrete-time noise covariance matrix but approximates the off-diagonal terms.

\begin{table}[h!]
	\centering
	\begin{tabular}[h]{|c|c|c|c|c|c|c|c|c|}
		\hline
		Quantity & 	$\lambda_a$ & $\sigma_a^2$ & $\lambda_{t_r}$ & $\sigma_{t_r}^2$ & $\sigma_{\beta_r}^2$ & $\lambda_b$ & $\sigma_b^2$ &  $\sigma_d^2$ \T \\ \hline
		value	 &  0.1 & 1.0 & 0.01 & 0.001 & 0 & & &  \T \\ \hline
		unit	 &  $s^{-1}$ & $s^{-1}$ &  $m^2s^{-5}$ & $m^2s^{-3}$ & $m^2s^{-3}$ &  & & \T \\ \hline				
	\end{tabular}
	\caption{Parameters of the continuous-time Markov process.}
	\label{table:error_std_prediction}
\end{table}