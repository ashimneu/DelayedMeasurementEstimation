In this section, position estimation using a linear Kalman filter will be designed using a position, velocity, acceleration (PVA) state space model.

\subsection{System Model}
The state estimation algorithm are designed using the standard state-space model:
\begin{align} \label{eqn:state_prop_model}
	\Boldx_{k+1} &\doteq \BoldF \Boldx_k + \BoldGamma \Boldw_k \\
	\label{eqn:msr_update_model}
	\Boldy_k &\doteq \BoldH_k \Boldx_k + \Boldeta_k \\
	\label{eqn:delayed_msr_update_model}
	\Boldz_k &\doteq \BoldC_{k-N} \Boldx_{k-N} + \Boldeta_k	
\end{align} 
The rover state vector is 
\begin{equation} \label{eqn:state_vector}
	\Boldx = [\,\Boldp^\top \!\!\!, \, \Boldv^\top \!\!\!, \, \Bolda^\top \!\!\!, \, N, \, b]^\top 
	\in \mathbb{R}^{n_s}
\end{equation}
where $\Boldp$, $\Boldv$, $\Bolda$ $\in \mathbb{R}^3$ represent the rover position, velocity and acceleration, respectively. $N$ is the delay time step of secondary measurement. $b$ is the secondary measurement bias. 
Therefore, $n_s = 11$. $\BoldF \in \mathbb{R}^{n_s \times n_s}$ is the state transition matrix.
$\Boldw_k \sim \mathcal{N}(\0,\,\BoldQ)$ is the process noise vector which is assumed to have a Gaussian distribution. $\BoldGamma \in \mathbb{R}^{n_s \times n_s}$ is the corresponding noise matrix. $\tau$ and $T$ are used to denote the time increments between successive state propagation and measurement update steps. 
Since the propagation is done at higher rate compared to measurement update, $\tau \ll T$.
When the GNSS pseudorange measurements are used the observation matrix $\BoldH$ is
\begin{equation}
	\BoldH = \begin{bmatrix}		
		\mathbfcal{H} & \0_{m \times 3} & \0_{m \times 3} & \1_{m \times 1} & \0_{m \times 1} \\		
	\end{bmatrix}
\end{equation}
where $\mathbfcal{H} = [\, \mathfrak{h}^1 \, \mathfrak{h}^2 \, \dots \, \mathfrak{h}^m ]^\top$ and each $\mathfrak{h}^s$ represent a line-of-sight unit vector between satellite $s$ and the rover GNSS antenna. 
The symbols $\0_{a \times b}$ and $\1_{a \times b}$ are  matrices with all entries $0$ and $1$, respectively, and with $a$ rows and $b$ columns. 
Similarly, $\BoldI_{a \times b}$ is an identity matrix with $a$ rows and $b$ columns.

\subsection{Kalman Filter Design}
The state space model has the standard form described in eqns. (\ref{eqn:state_prop_model}-\ref{eqn:msr_update_model}). 
The rover state vector is defined as
$\Boldx_v = [\,\Boldp^\top \!\!\!\!,  \, \Boldv^\top \!\!\!\!, \, \Bolda^\top]^\top$ 
and the receiver clock state vector is defined as $\Boldx_c = [\,N, \, b]^\top$.
The sub-components $\Boldx_v$ and $\Boldx_c$ propagate independently through time hence their respective components in the state space model are not coupled.
The matrices of the discrete-time state-space model are
\begin{align}
	\BoldF = \begin{bmatrix} \BoldF_v & \0 \\ \0 & \BoldF_c \end{bmatrix}, ~
	\BoldGamma = \begin{bmatrix} \BoldGamma_v & \0 \\ \0 & \BoldGamma_c	\end{bmatrix}, ~
	\BoldQ = \begin{bmatrix} \BoldQ_v & \0 \\ \0 & \BoldQ_c \end{bmatrix}
\end{align}
%The time step of state propagation is defined as $T = 0.1$ seconds and the measurement update using GNSS pseudorange occur once every 1 second.

The continuous-time PVA vehicle model is
\begin{align}
	\dot{\Boldx}_v(t) \doteq 
	\begin{bmatrix}
		\0 & \BoldI & \0 \\ \0 & \0 & \BoldI \\ \0 & \0 & -\lambda_a \BoldI
	\end{bmatrix} 
	\Boldx_v(t) +
	\begin{bmatrix} \0 & \0 & \0 \\ \0 & \0 & \0 \\ \0 & \0 & \BoldI \end{bmatrix} \Boldw_a(t)
\end{align}
where $\Boldomega_a(t)$ is modeled as Gaussian white noise with power spectral density $\BoldP_a = \sigma^2_a$. 
The corresponding discrete-time description of the PVA vehicle model is approximated as
\begin{align}
	\BoldF_v &= 
	\begin{bmatrix}
		\BoldI_{3 \times 3} & T\,\BoldI_{3 \times 3} & a_3 \, \BoldI_{3 \times 3} \\
		\0_{3 \times 3} & \BoldI_{3 \times 3} & a_2 \, \BoldI_{3 \times 3} \\
		\0_{3 \times 3} & \0_{3 \times 3} & a_1 \, \BoldI_{3 \times 3}
	\end{bmatrix}, ~ \\
	\BoldGamma_v &\approx
	\begin{bmatrix}
		\big(T^5/20 \big)^{\frac{1}{2}} \, \BoldI_{3 \times 3} & \0_{3 \times 3} & \0_{3 \times 3}\\
		\0_{3 \times 3} & \big(T^3 / 3\big)^{\frac{1}{2}} \, \BoldI_{3 \times 3} & \0_{3 \times 3} \\
		\0_{3 \times 3} & \0_{3 \times 3} & \sqrt{T} \, \BoldI_{3 \times 3}
	\end{bmatrix}, \\ 
	\BoldQ_{v} &= 
	\begin{bmatrix} \0_{3 \times 3} & \0_{3 \times 3} & \0_{3 \times 3} \\ 
					\0_{3 \times 3} & \0_{3 \times 3} & \0_{3 \times 3} \\
		 			\0_{3 \times 3} & \0_{3 \times 3} & \sigma^2_a \, \BoldI_{3 \times 3} 
	\end{bmatrix},
\end{align}
where $a_1 = \exp^{\lambda_a T}$, $a_2 = (1 - \exp^{-\lambda_a T})/ \lambda_a$, and $a_3 = (\lambda_a T - 1 + \exp^{-\lambda_a T})/ \lambda_a^2$.