The solution approach described in Section \ref{sect:estimation_delayed_msr} is proposed for position estimation of a trolley mounted on a gantry crane at a container yard (shown in Figure \ref{fig:field_front_view}). 
The schematic digram in Figure \ref{fig:field_front_view} illustrates the physical setup of the gantry crane in a yard above a row of containers. 
Similarly, Figure \ref{fig:field_side_view} illustrates a side view of the yard.  
Coordinate axes of various local reference frames have been labelled. 
The relevant reference frames are Field Frame (F), Gantry Frame (G) and Trolley Frame (T).

A trolley traverses parallel to $G_x$ axis, on a rail placed on top of the gantry structre. The gantry-crane structure can travel along $F_y$ axis.

The position of the trolley is measured using two sensors: a Global Navigation Satellite System (GNSS) receiver and an encoder. 
The GNSS receiver is permanently place on top of the trolley (interchangeably also called rover) and collects raw measurements from positioning satellites to implement differential positioning.
The encoder is situated on the gantry rail and measures position of the trolley along the $G_x$ direction.
